%% It is just an empty TeX file.



\documentclass{article}

\usepackage{graphicx}

\usepackage{pdfpages}

\title{Salsa20} 


\begin{document}             % End of preamble and beginning of text.

\maketitle                   % Produces the title.



\section{How it is formed}

 

%%\includegraphics[scale=.1]{dark-wallpaper-36.jpg}


Salsa20 is a counter-based stream cipher—it generates its keystream by
repeatedly processing a counter incremented for each block. As you can
see in Figure 5-10, the Salsa20 core algorithm transforms a 512-bit block
using a key (K), a nonce (N), and a counter value (Ctr). Salsa20 then adds
the result to the original value of the block to produce a keystream block.
(If the algorithm were to return the core’s permutation directly as an
output, Salsa20 would be totally insecure, because it could be inverted.
The final addition of the initial secret state K , N , Ctr makes the
transform key-to-keystream-block non-invertible.)




\includegraphics[scale=1]{img2.png}

(sorry for the foreign language; coudn't find a picture in english but that 
sentence means SALSA20 CORE): REFERENCE IN THE BOOK: SERIOUS CRYPTOGRAPHY(CHAPTER 5) 


%%somethinghere

\includegraphics[scale=.5]{Mytitle2_page-0001.jpg}



\subsection{observation from the page taken from notes}


This is a matrix of 4x4 word(4 bytes each);
meaning that each block give a keystream of 64 bytes(4*4*4).
That is why we have to take (b/64) to get the total number of blocks needed to
make a valid key stream for a plaintext of b bits.




\section{The quarter-round function}


Salsa20’s core permutation uses a function called quarter-round (QR) to
transform four 32-bit words (a, b, c, and d), as shown here:


   b=b xor [(a+d) \(<<<\) 7]
   
   
   c=c xor [(b+a) \(<<<\) 9]
   
   
   d=d xor [(c+b) \(<<<\) 13]
   
   
   a=a xor [(d+c) \(<<<\) 18]

These four lines are computed from top to bottom, meaning that the
new value of b depends on a and d, the new value of c depends on a and on
the new value of b (and thus d as well), and so on.
The operation \(<<<\) is wordwise left-rotation by the specified number of
bits, which can be any value between 1 and 31 (for 32-bit words).



\includegraphics[scale=.5]{My3title.jpg}

\subsection{Observation from the teacher and students}


1)In case of the syncronized mode,how is the IV initialized


Answer:It depends; u can make it public if u wanted; it depends on what you choose or u can even choose some random thing and share it.


2)Key whiting: it is the addition done at the end of the quarter-round
function and the key; since if we don't do it; it would be reversible 
hence insecure.


3)the meaning of (a,b,c,d) in the matrix:
Prof's answer: It means that when we pass those 4 numbers in the qr function; they are going to be updated according to the schema above. 


4)with a certain number of registers each with a part of the stream with bytes in it; in what order do we xor with the plain text?
Prof answer: In the bernstein paper, they had in mind the little endian notation in the lower level but on higher level like in python, it depends on how u implemented the bytes(for example in python u use lists)

5)where does the rotation number coming from(7,9,13,18):
Answer: They are some ratio in the berstein paper to some explanation they used those numbers

6)Salsa works on plaintext of variable size because only the number of registers change

7)what do we do at decryption side?:
answer: You just do the same; no need of the reverse since it is just a
xor with the cyphertext.

\subsection{Salsa20 implementation example in python}

from salsa20 import Xsalsa20\(_\)xor


from os import urandom


IV = urandom(24)


KEY = b'*secret**secret**secret**secret*'


ciphertext = XSalsa20\(_\)xor(b"IT'S A YELLOW SUBMARINE", IV, KEY)


print(ciphertext)


print(XSalsa20\(_\)xor(ciphertext, IV, KEY).decode())
 
%% \includepdf{}


\section{CHACHA20}

The only difference between chacha20 is the quarter-round.

\includegraphics[scale=.5]{Mytitle9.jpg}

\subsubsection{what is added}

confusion: (Shanon theory) It is related with permutation; (This refers to change the position of values in the register)

difusion: register and matrix; the register multiplied with matrix; (this is a way of calculating the media since the matrix perse is populated with 1s and 0s)







\end{document}   
%% Write your code here.